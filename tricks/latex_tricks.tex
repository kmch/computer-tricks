
% New line after paragraph.
\paragraph{Equation of motion}~\\ This text will be in a new line.

% partial derivatives, (see LaTeX_my_templates for more) 
\newcommand{\pder}[2][]{\frac{\partial#1}{\partial#2}}
% dy/dx:
\pder[y]{x}
% d/dx:
\pder{x}
%%


% line numbering 
\usepackage{lineno}
\linenumbers
%%

% LISTING FORTRAN CODE WITH SYNTAX-COLORING
\usepackage{listings}
\lstset{language=[90]Fortran,
  basicstyle=\ttfamily,
  keywordstyle=\color{red},
  commentstyle=\color{green},
  morecomment=[l]{!\ }% Comment only with space after !
}
%%


% ASTRONOMICAL SYMBOLS
\usepackage{mathabx} 
$\Earth$
%%

% LAGRANGIAN SYMBOL
\newcommand{\Lagr}{\mathcal{L}} 
%% 

% ELECTRIC CIRCUITS SYMBOLS
\usepackage{circuitikz} 
\begin{document}
\ctikzset{bipoles/length=.6cm}
\newcommand\esymbol[1]{\begin{circuitikz}
\draw (0,0) to [#1] (1,0); \end{circuitikz}}
% CAN'T BE PUT IN THE SECTION HEADER
\esymbol{generic} % RESISTOR 
% ...
%%

% FULL-WIDTH FIGURES IN A 2-COLUMN DOC
\begin{figure*}
\end{figure*}
%%

% DOCUMENT WITH DIFFERENCES
sudo apt-get install latexdiff
latexdiff draft.tex revision.tex > diff.tex
latexdiff -t CTRADITIONAL draft.tex revision.tex > diff.tex
%     UNDERLINE – Added text is wavy-underlined and blue, discarded text is struck out and red
%     CTRADITIONAL – Added text is blue and set in sans-serif, and a red footnote is created for each discarded piece of text
%     TRADITIONAL – Like CTRADITIONAL but without the use of color
%     CFONT – Added text is blue and set in sans-serif, and discarded text is red and very small size
%     FONTSTRIKE – Added text is set in sans-serif, discarded text small and struck out
%     CHANGEBAR – Added text is blue, and discarded text is red. Additionally, the changed text is marked with a bar in the margin


%%

%% POPRAWNOŚĆ JĘZYKOWA

% Znaki diakrytyczne:
\k{a} %- ą/ę
\c a  %- ą jak nie działa powyższe (ale TO NIE JEST poprawny ogonek)
\'s   %- ś, ó, ź, Ń
\.z   %- ż
\l    %- ł, Ł
\"u   %- u umlaut i inne
\`e   %- jak w Amp\`ere (klawisz z tyldą)
\`    %  (grave accent): à
\'    %  (acute accent): á
\^    %  (circumflex or “hat”): â
\"    %  (umlaut or dieresis): ä
\~    %  (tilde or “squiggle”): ã
\=    %  (macron or “bar”): ā
\.    %  (dot accent): ȧ
\u    %  (breve accent): ă
\v    %  (háček or “check”): ǎ
\H    %  (long Hungarian umlaut): ő
\t    %  (tie-after accent): a͡
\c    %  (cedilla): ş
\d    %  (dot-under accent): ạ
\b    %  (bar-under accent): ο̩
\k    %  (ogonek): ą

%%

% myślnik, półpauza, dywiz, minus
% myślnik
\LaTeX --- moja miłość
% myślnik w angielskim (bez spacji)
\LaTeX---moja miłość
% lub bardziej estetycznie (krócej), ale purysta by się obraził:
\LaTeX -- moja miłość 
\LaTeX--moja miłość
% półpauza (zakresy) - bez odstępów, jeśli otaczają ją liczby:
str. 11--13
1935--2016
% a jeśli nie, to z odstępami:
3 marca -- 2 października
% dywiz (łącznik) - łączy wyrazy wieloczłonowe:
niebiesko-czarni
% uwaga, polskie dzielenie wyrazów każe:
niebiesko-
-czarni
% żeby to uzyskać:
- pakiet ,,platex'' do j. polskiego
- zapis: niebiesko{\dywiz}czarni
% minus
$-2$
%%

% ułamek dziesiętny w polskiej typografii
$0{,}5$
% lub:
\usepackage{icomma} % turns off unwanted space after comma in math mode
$0,5$
%%

% quotation marks (cudzysłów)
``quotation''
% W języku polskim cudzysłów otwierający oznacza się dwoma przecinkami ,, , natomiast zamykający – dwoma apostrofami ’’.
,,cytat''
%%

%%%




%% INNE

% when \newpage is ignored (TeX thinks it's already empty)
\null\newpage
% to suppress showing the number (but it's included into numbering)
\thispagestyle{empty}
\null\newpage % in this order!!!
%%

% mailto
\href{mailto:mwilde@igf.fuw.edu.pl}{mwilde@igf.fuw.edu.pl}
%%

%
\chapter[toc version]{doc version}
\chaptermark{version for header (usually shortened)}
%%

%
\begin{itemize}
  \item [] empty bullet
\end{itemize}
%%

% lorem ipsum
\usepackage{blindtext}
\Blindtext[2][3] % where 2 (here) means number of paragraphs, and 3 (here) the measures the length of each paragraph
%%


%%%




%% TABELE
% Tabela instead of tablica
\renewcommand*{\tablename}{Tabela}
%%

% change of fontsize in table
\tiny % right after \begin{table}
%%

%%%


%% SPISY
% prevent ToC from appearing in... Toc
\begin{KeepFromToc}
Spis Treści
\end{KeepFromToc}
%%

% add to ToC chapter*, section* etc. 
Wprowadzenie <<Chapter*>> % this line is as usual
\addcontentsline{toc}{chapter}{\numberline{}Wprowadzenie} % this special line follows
%%

%%%



%% MATEMATYKA

% pogrubienie w trybie matematycznym
$boldsymbol{v}$
%%

% granice całki \int albo sumy \sum NAD symbolem, a nie obok (zwykłe _ i ^)
\frac{\int\limits_{1}^{2}R\left|y\right|dx}{l_m}
%%

% Newton symbol
{N}\choose{k}
% or in case of problems
{{N}\choose{k}}
%%

% end of proof symbol
\usepackage{amsthm}
\begin{proof}
\end{proof}
%%

%
\bm bold math (can denote vectors instead of arrow - \vec)
%%

%
\vec vector
\ge         %- 'greater than or equal to' sign
\le         %- 'less than or equal to' sign
\permil     %- permil (promil)
\neq        %- znak 'nie równa się'
\Rightarrow %- implikacja
\ast        %- gwiazdka (splot)

%%

% double-line letters (sets of numbers style)
\mathbb{R}
% another fancy style
\mathcal{R}
%%

%%%




%
\lettrine[lines=2]{\color{MidnightBlue}W}{}
%%

% fancy colors
NavyBlue
RoyalBlue
MidnightBlue
BlueViolet
\def\cthesissetcolorbluemagenta{%
	\cthesissetcolor{cmyk}{1, .50, .10, .01}{.18, .98, .18, 0}%
}

% sets the blue-green color theme (blue/green)
\def\cthesissetcolorbluegreen{%
	\cthesissetcolor{cmyk}{.61, .47, .03, 0}{.48, .05, .91, 0}%
}
%%








%%


% uninstall MikTeX on Windows
I suppose you have a standard installation. Go to C:\Program Files\MiKTeX 2.9\miktex\bin\x64\internal\. 
Right-click on uninstall_admin.exe and select Execute as Administrator
%%
%
\int\limits_{-\infty}^{+\infty}
%%

% komentowanie wielu linii
\iffalse
I don't want this to appear
\fi
%%





Spaces after commands without braces get suppressed (ignorowane, tłumione)
%Example:
%Copyright \copyright 2013
To  prevent  this,  put  empty  curly  braces  after  the  command.
%Example:
%Copyright \copyright{} 2013


Packages like longtable and array
can help with
more complex table formats

spacja niełamliwa to tylda ~

\mbox{long-piece-of-text}   - prevent from breaking the word

\begin{itemize}
\item[] kkhj % empty bullets
\end{itemize}


No number on title page:
\maketitle sets the page style to plain, so you need to move \thispagestyle{empty} to after \maketitle:




One should use \[ \] instead of $ $ for math mode, because latex doesn't support the latter!
% \[ is a short form of \begin{displaymath} which one might expect to act like an un-numbered form of \begin{equation}. The amsmath package redefines \[ to be \begin{equation*} which is exactly an un-numbered form of the equation environment as defined by that package. In the core LaTeX definition \[ has the definition

% directory to copy own or manually downloaded .sty file to. After that usually one needs to run:
/usr/share/texmf/tex/latex  
% to update the TeX's directory tree.
sudo texhash

% OR CREATE:
~/texmf/tex/latex
% TO PUT .cls AND .sty FILES THERE AND
/home/kmc3817/texmf/bibtex
% TO PUT .bst FILES THERE 
% THEN JUST TYPE 
texhash ~/texmf
% YOU CAN CHECK IF THE FILES ARE SEEN BY TEX BY:
kpsewhich filename.sty
kpsewhich filename.cls
kpsewhich filename.bst
% THIS SHOULD PRINT ABSOLUTE PATHS TO THEM