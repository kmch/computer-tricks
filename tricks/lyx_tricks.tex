% Dark mode (Seti black, https://wiki.lyx.org/Tips/ColorSchemes)
% In C:\Users\Maciek\AppData\Roaming\LyX2.3\preferences
% (this folder is HIDDEN and can't be access from LyX 'open' tool. Use Explorer instead)
% paste:
\set_color "cursor" "#A6E22E"
\set_color "background" "#151718"
\set_color "foreground" "#d4d7d6"
\set_color "selection" "#4FA5C7"
\set_color "selectiontext" "#fefefe"
\set_color "latex" "#A6E22E"
\set_color "preview" "#A6E22E"
\set_color "inlinecompletion" "#AE81FF"
\set_color "nonuniqueinlinecompletion" "#AE81FF"
\set_color "note" "#75715E"
\set_color "notebg" "#252728"
\set_color "commentbg" "#252728"
\set_color "greyedoutbg" "#252728"
\set_color "shaded" "#dc322f"
\set_color "listingsbg" "#75715E"
\set_color "footlabel" "#AE81FF"
\set_color "urllabel" "#A6E22E"
\set_color "urltext" "#A6E22E"
\set_color "depthbar" "#A6E22E"
\set_color "language" "#A6E22E"
\set_color "command" "#A6E22E"
\set_color "commandbg" "#252728"
\set_color "commandframe" "#AE81FF"
\set_color "special" "#66D9EF"
\set_color "graphicsbg" "#252728"
\set_color "math" "#A6E22E"
\set_color "mathbg" "#252728"
\set_color "mathmacrobg" "#252728"
\set_color "mathmacroframe" "#A6E22E"
\set_color "mathmacroblend" "#A6E22E"
\set_color "mathmacronewarg" "#A6E22E"
\set_color "mathframe" "#A6E22E"
\set_color "mathcorners" "#A6E22E"
\set_color "mathline" "#A6E22E"
\set_color "collapsible" "#d0caff"
\set_color "collapsibleframe" "#A6E22E"
\set_color "insetframe" "#AE81FF"
\set_color "eolmarker" "#A6E22E"
\set_color "added_space" "#A6E22E"
\set_color "appendix" "#dc322f"
\set_color "changebar" "#4d4d4c"
\set_color "addedtext" "#8b0000"
\set_color "changedtextauthor1" "#A6E22E"
\set_color "changedtextauthor2" "#d33682"
\set_color "changedtextauthor3" "#cf4b16"
\set_color "changedtextauthor4" "#b58900"
\set_color "changedtextauthor5" "#859900"
\set_color "tabularline" "#A6E22E"
\set_color "tabularonoffline" "#AE81FF"
\set_color "newpage" "#A6E22E"
\set_color "pagebreak" "#75715E"
\set_color "buttonframe" "#AE81FF"
\set_color "buttonbg" "#252728"
\set_color "buttonhoverbg" "#454748"
\set_color "paragraphmarker" "#8abeb7"
\set_color "previewframe" "#A6E22E"
%%

%
on hp, ctrl+t for pdf output
%%

% Restoring the default bind file in LyX
% (Unable to find bind file)
1. Open LyX and go to About LyX. Write down your User directory (this should be copyable in the next version).
2. Exit LyX
3. Navigate to where your User directory is and delete it (the whole directory!). Actully, just move it so that you can restore it if necessarys.
4. Restart LyX
%%

% In some cases LaTeX does not know how to hyphenate a word, and then it can be helped along by explicitly marking up 
% hyphenation points. In LaTeX code this is done by \-, e.g. (3-glycid\-oxy\-propyl)\-tri\meth\-oxy....
% In LyX you do this with Insert --> Formatting --> Hyphenation point (keyboard shortcut Ctrl + -).
%%

%
It's pretty straightforward to install newest version of LyX from source even if on outdated distro. Just follow 
README and INSTALL
%%


%% SPISY

% add chapter*, section* etc. to ToC
Wprowadzenie <<Chapter*>> % this line is as usual
\addcontentsline{toc}{chapter}{\numberline{}Wprowadzenie} % this special line follows
%%

%%%




%% BIBLIO

% errors of bibtex are in different tab than lyx errors (shown output)
%%

% use a main bibliography for the whole multi-part document, but separate bibliographies when chapters are compiled separately
% 1) For the main document, you just insert a bibliography inset at the place where the main bibliography has to appear (within the master file or within a child).
% 2) Within the child documents, insert bibliography insets where the bibliography should appear when the child is compiled separately, but insert them into a branch (Insert→Branch→Insert New Branch...), e.g. called "Childonly".
% 3) Within the children, activate the branch (Document→Settings...→Branches).
% 4) Within the master, de-activate the branch (Document→Settings...→Branches). 
%%

%%%




%% WYPEŁNIENIE STRONY

% marginesy - najlepiej je ustawić w ustawieniach dokumentu (->marginesy):
wewnętrzny    3 cm
zewnętrzny    2.5 cm
odstęp stopki 1.5 cm
% Reszta pól niewypełniona
%%

% żeby na stronie POZIOMEJ (\begin{landscape}) obrazek wypełniał całą stronę
ustaw szerokość: 100 procent szerokości strony
%%

% 47% szerokości tekstu (jeśli 2 obrazki obok siebie)
% jeśli różna wysokość (ale oba kwadratowe) to po 7 cm wysokości
%%

% kolumnowy układ subfigur:
1) wstawka->rysunek
2) do niej wstaw tyle wstawek-rysunków (dla nich się nie określa szerokości), ile kolumn ma być
3) do każdej tej sub-wstawki:
- użyj wstaw->box->bezramkowe (np. 47\%)
- wstaw do boksa całą kolumnę obrazków - każdy z nich powinien mieć 100\% SZEROKOŚCI LINII!
%%

%%%




%% SUBDOKUMENTY

% listing kodu
Wstaw->Plik->Dokument podrzędny
Typ wstawienia: listing kodu
Więcej parametrów:
breaklines=true % łamie linie, żeby nie wychodziły za marginesy
language=Python % koloruje składnie (np.) Pythona
captionpos=b % pozycja tytułu?
frame=tb % położenie na stronie?
%%

%%%




%% MATH MODE

%
sekwencja: alt+m, g, a wpisuje w trybie matematycznym (ctrl+m) alfę. Analogicznie inne greckie litery! 
%%

% Prevent LaTeX from inserting a linebreak in an inline formula (ctrl+m):
1. Enter to math mode (either for inline formula or display formula)
2. Type \mbox and then a press ENTER or SPACE
3. The text "\mbox" disappears and a blue box appears in its place
4. Select math mode (again!)
5. Type the equation inside the box 
%%

%
ctrl+space  - space in math environment
%%

%%%


%% INSTALACJA, KONFIGURACJA, ZARZĄDZANIE PROJEKTEM

% It's important to have main.lyx in the "most-parent" directory
% with regard to all other files e.g. in content/.
% Thereby ALL PATHS ARE RELATIVE and you can move the whole project 
% somewhere else without necessity of rewriting paths
%%

% instalacja lyx > 2.2 na windowsie
miktex musi być w wersji 32bit! Lyx: trzeba sciagnac installer 3 (chyba, ale może miktex załatwia sprawę)
%%

%%%
