%% NATBIB

% FOR SPRINGER JOURNALS I HAD TO ADD:
\usepackage[sort&compress]{natbib}
% TO MAKE WORK THE FOLLOWING STYLE:
\bibliographystyle{spbasic} % basic style, author-year citations
%%


% opcje natbiba
%Opcja 		Znaczenie
round	 	%Nawiasy () (domyślne)
square	 	%Nawiasy kwadratowe []
curly 		%Curly braces {}
angle 		%nawiasy ostrokątne <>
colon	 	%wielokrotne odwołania oddzielone są średnikami (domyślnie)
comma 		%wielokrotne odwołania oddzielone są przecinkami
authoryear 	%styk autor-rok (domyślny)
numbers 	%zwykła numeracja
super 		%superscripted numeric citations
sort 		%wielokrotne odwołania sortowanę są zgodnie z występowaniem w bibliografii
sort&compress 	%jak wyżej + kompresowanie wielokrotnych pozycji w bibliografii
longnamesfirst 	%pierwsze odwołanie do każdej pozycji w bibliografii spowoduje wypisanie wszystkich autorów, kolejne odwołania po pierwszym nazwisku będą miały wpisane et al.
sectionbib 	%używane z pakietem chapterbib. zmienia \chapter* na \section*
nonamebreak 	%powoduje utrzymanie listy nazwisk autorów w odnośniku w jednej linii
%% 

%
\bibpunct{(}{)}{;}{a}{,}{,}
%     Symbol otwieranego nawiasu.
%     Symbol zamykanego nawiasu.
%     Symbol jaki pojawi się między wielokrotnymi odwołaniami.
%     Jedna z trzech możliwości:
%         n - styl numeryczny.
%         s - inny styl numeryczny.
%         inna litera - styl autor-rok.
%     Znak pojawiający się między nazwiskiem autora a rokiem (tylko w przypadku nawiasowym).
%     Znak pojawiający się między latami, w przypadku wielokrotnego odniesienia do artykułów napisanych przez jednego autora, np. (Chomsky 1956, 1957). Aby otrzymać dodatkową spację, należy użyć komendy {,~}.
%%



% tekst przed i po, wewnątrz nawiasów
\citep[e.g.][page 3]{Foster2014} 
%%

%
\citet{goossens93} 	Goossens et al. (1993) % 't' for textual
\citep{goossens93} 	(Goossens et al., 1993) % 'p' for parantheses
\citet*{goossens93} 	Goossens, Mittlebach, and Samarin (1993)
\citep*{goossens93} 	(Goossens, Mittlebach, and Samarin, 1993)
\citeauthor{goossens93} Goossens et al.
\citeauthor*{goossens93}Goossens, Mittlebach, and Samarin
\citeyear{goossens93} 	1993
\citeyearpar{goossens93}(1993)
\citealt{goossens93} 	Goossens et al. 1993
\citealp{goossens93} 	Goossens et al., 1993
%%


%%%

% zmiana et al. na i in., wyedytuj plik .bst w miejscu:
FUNCTION {bbl.etal}
%%


% Cytowanie www
@misc{WinNT,
  title = {{MS Windows NT} Kernel Description},
  howpublished = {\url{http://web.archive.org/web/20080207010024/http://www.808multimedia.com/winnt/kernel.htm}},
  note = {Accessed: 2010-09-30}
}
%%

% zastąpienie słowa 'and' przecinkiem.
% W pliku 
unsrturl.bst 
% z katalogu 
/usr/share/texmf-texlive/bibtex/bst/urlbst 
% wyedytowałem następujące linie (docelowa postać podana na prawo od numeru):
336 		{ "" * }
341 		{ ", " * t * }
619 		{ ", " * editor #2 "{vv~}{ll}" format.name$ * }
%%


% Lic. z fizyki:
% STYL: 
myunsrt
%DOMYŚLNY STYL:
\use_bibtopic false
%%

% W pliku .bib jeśli zapomni się przecinka po rekordzie, to ostatni rekord nie będzie już wyświetlony! 
% Brak przecinka jest dozwolony więc tylko po ostatnim rekordzie. 
%%


% Włączenie bazy bibliograficznej w pliku następuje poprzez użycie polecenia bibliography: 
\bibliography{plik} 
% gdzie plik to nazwa pliku z bazą)
%%


% Ogólny podział na 4 style cytowania:
    plain - pozycje bibliograficzne są posortowane alfabetycznie i ponumerowane,
    unsrt - pozycje bibliograficzne występują w kolejności cytowania i są ponumerowane [1], [2], ...
    alpha - pozycje są posortowane, ale zamiast numerów mają etykiety typu Gu96,
    abbrv - format podobny do plain, ale imiona autorów, nazwy miesięcy i nazwy czasopism są skracane.
%%


% program MAKEBST (składnik pakietu custombib)
% tryb interaktywny
latex makebst
% tryb wsadowy - szybciej wyedytować zrobiony wcześniej szablon makebst_default.dbj a następnie w konsoli wpisać:
tex plik.dbj
% co wygeneruje plik.bst, który można wczytać w LyX, klikając w bibliografię i wybierając plik stylu.

% UWAGA, ten styl 'jest w stylu' stylu 'plain' (author-year). 
% Wymaga pakietu natbib, należy wpisać w preambule pliku LaTeXowego:
\usepackage[round]{natbib} % round for parantheses instead of brackets
\renewcommand{\cite}{
   \citep 
} % bo domyślnie LyX wstawia cytowanie poprzez komendę \cite, a to daje 
% gorszy efekt (można zresztą to zahaszować i zobaczyć

% Ważne linie do wyedytowania

% separation of key words (blocks/section, e.g. title and journal)
%PUNCTUATION BETWEEN SECTIONS (BLOCKS):
blk-com,%: Comma between blocks
%%